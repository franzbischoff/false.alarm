% This is the Reed College LaTeX thesis template. Most of the work
% for the document class was done by Sam Noble (SN), as well as this
% template. Later comments etc. by Ben Salzberg (BTS). Additional
% restructuring and APA support by Jess Youngberg (JY).
% Your comments and suggestions are more than welcome; please email
% them to cus@reed.edu
%
% See https://www.reed.edu/cis/help/LaTeX/index.html for help. There are a
% great bunch of help pages there, with notes on
% getting started, bibtex, etc. Go there and read it if you're not
% already familiar with LaTeX.
%
% Any line that starts with a percent symbol is a comment.
% They won't show up in the document, and are useful for notes
% to yourself and explaining commands.
% Commenting also removes a line from the document;
% very handy for troubleshooting problems. -BTS

% As far as I know, this follows the requirements laid out in
% the 2002-2003 Senior Handbook. Ask a librarian to check the
% document before binding. -SN

%%
%% Preamble
%%
% \documentclass{<something>} must begin each LaTeX document
\documentclass[12pt,twoside]{reedthesis}
% Packages are extensions to the basic LaTeX functions. Whatever you
% want to typeset, there is probably a package out there for it.
% Chemistry (chemtex), screenplays, you name it.
% Check out CTAN to see: https://www.ctan.org/
%%
\usepackage{graphicx,latexsym}
\usepackage{amsmath}
\usepackage{amssymb,amsthm}
\usepackage{longtable,booktabs,setspace}
\usepackage{chemarr} %% Useful for one reaction arrow, useless if you're not a chem major
\usepackage[hyphens]{url}
% Added by CII
\usepackage{hyperref}
\usepackage{lmodern}
\usepackage{float}
\floatplacement{figure}{H}
% Thanks, @Xyv
\usepackage{calc}
% End of CII addition
\usepackage{rotating}

% Next line commented out by CII
%%% \usepackage{natbib}
% Comment out the natbib line above and uncomment the following two lines to use the new
% biblatex-chicago style, for Chicago A. Also make some changes at the end where the
% bibliography is included.
%\usepackage{biblatex-chicago}
%\bibliography{thesis}


% Added by CII (Thanks, Hadley!)
% Use ref for internal links
\renewcommand{\hyperref}[2][???]{\autoref{#1}}
\def\chapterautorefname{Chapter}
\def\sectionautorefname{Section}
\def\subsectionautorefname{Subsection}
% End of CII addition

% Added by CII
\usepackage{caption}
\captionsetup{width=5in}
% End of CII addition

% \usepackage{times} % other fonts are available like times, bookman, charter, palatino

% Syntax highlighting #22

% To pass between YAML and LaTeX the dollar signs are added by CII
\title{Detecting life-threatening patterns in Point-of-care ECG using efficient memory and processor power}
\author{Francisco Bischoff}
% The month and year that you submit your FINAL draft TO THE LIBRARY (May or December)
\date{Jul 2020}
\division{CINTESIS}
\advisor{Pedro Pereira Rodrigues}
\institution{Faculdade de Medicina da Universidade do Porto}
\degree{Ph.D.~in Health Data Science}
%If you have two advisors for some reason, you can use the following
% Uncommented out by CII
\altadvisor{Eamonn Keogh}
% End of CII addition

%%% Remember to use the correct department!
\department{Medical Investigation}
% if you're writing a thesis in an interdisciplinary major,
% uncomment the line below and change the text as appropriate.
% check the Senior Handbook if unsure.
%\thedivisionof{The Established Interdisciplinary Committee for}
% if you want the approval page to say "Approved for the Committee",
% uncomment the next line
%\approvedforthe{Committee}

% Added by CII
%%% Copied from knitr
%% maxwidth is the original width if it's less than linewidth
%% otherwise use linewidth (to make sure the graphics do not exceed the margin)
\makeatletter
\def\maxwidth{ %
  \ifdim\Gin@nat@width>\linewidth
    \linewidth
  \else
    \Gin@nat@width
  \fi
}
\makeatother

% From {rticles}
\newlength{\csllabelwidth}
\setlength{\csllabelwidth}{3em}
\newlength{\cslhangindent}
\setlength{\cslhangindent}{1.5em}
% for Pandoc 2.8 to 2.10.1
\newenvironment{cslreferences}%
  {}%
  {\par}
% For Pandoc 2.11+
% As noted by @mirh [2] is needed instead of [3] for 2.12
\newenvironment{CSLReferences}[2] % #1 hanging-ident, #2 entry spacing
 {% don't indent paragraphs
  \setlength{\parindent}{0pt}
  % turn on hanging indent if param 1 is 1
  \ifodd #1 \everypar{\setlength{\hangindent}{\cslhangindent}}\ignorespaces\fi
  % set entry spacing
  \ifnum #2 > 0
  \setlength{\parskip}{#2\baselineskip}
  \fi
 }%
 {}
\usepackage{calc} % for calculating minipage widths
\newcommand{\CSLBlock}[1]{#1\hfill\break}
\newcommand{\CSLLeftMargin}[1]{\parbox[t]{\csllabelwidth}{#1}}
\newcommand{\CSLRightInline}[1]{\parbox[t]{\linewidth - \csllabelwidth}{#1}}
\newcommand{\CSLIndent}[1]{\hspace{\cslhangindent}#1}

\renewcommand{\contentsname}{Table of Contents}
% End of CII addition

\setlength{\parskip}{0pt}

% Added by CII

\providecommand{\tightlist}{%
  \setlength{\itemsep}{0pt}\setlength{\parskip}{0pt}}

\Acknowledgements{

}

\Dedication{

}

\Preface{

}

\Abstract{

}

	\usepackage{booktabs}
 \usepackage{longtable}
 \usepackage{array}
 \usepackage{multirow}
 \usepackage{wrapfig}
 \usepackage{float}
 \usepackage{colortbl}
 \usepackage{pdflscape}
 \usepackage{tabu}
 \usepackage{threeparttable}
 \usepackage{threeparttablex}
 \usepackage[normalem]{ulem}
 \usepackage{makecell}
 \usepackage{xcolor}
% End of CII addition
%%
%% End Preamble
%%
%
\begin{document}

% Everything below added by CII
  \maketitle

\frontmatter % this stuff will be roman-numbered
\pagestyle{empty} % this removes page numbers from the frontmatter



  \hypersetup{linkcolor=black}
  \setcounter{secnumdepth}{2}
  \setcounter{tocdepth}{2}
  \tableofcontents

  \listoftables

  \listoffigures



\mainmatter % here the regular arabic numbering starts
\pagestyle{fancyplain} % turns page numbering back on

\hypertarget{acknowledgements}{%
\section*{Acknowledgements}\label{acknowledgements}}
\addcontentsline{toc}{section}{Acknowledgements}

I want to thank a few people.

\hypertarget{preface}{%
\section*{Preface}\label{preface}}
\addcontentsline{toc}{section}{Preface}

This is an example of a thesis setup to use the reed thesis document class (for LaTeX) and the R bookdown package, in
general.

\hypertarget{dedication}{%
\section*{Dedication}\label{dedication}}
\addcontentsline{toc}{section}{Dedication}

You can have a dedication here if you wish.

\hypertarget{abstract}{%
\section*{Abstract}\label{abstract}}
\addcontentsline{toc}{section}{Abstract}

Currently, Point-of-Care (POC) ECG monitoring works either as plot devices or alarms for abnormal cardiac rhythms using
predefined normal trigger ranges and some rhythm analysis, which raises the problem of false alarms. In comparison,
complex 12-derivation ECG machines are not suitable to use as simple monitors and are used with strict techniques for
formal diagnostics. We aim to identify, on streaming data, life-threatening hearth electric patterns to reduce the
number of false alarms, using low CPU and memory maintaining robustness. The study design is comparable to a diagnostic
study, where high accuracy is essential. Physionet's 2015 challenge yielded very good algorithms for reducing false
alarms. However, none of the authors reported benchmarks, memory usage, robustness test, or context invariance that
could assure its implementation on real monitors to reduce alarm fatigue indeed. We expect to identify the obstacles of
detecting life-threatening ECG changes within memory, space, and CPU constraints and to reduce ECG monitor's false
alarms using the proposed methodology, and assess the feasibility of implementing the algorithm in the real world and
other settings than ICU monitors.

\hypertarget{introduction}{%
\chapter*{Introduction}\label{introduction}}
\addcontentsline{toc}{chapter}{Introduction}

Currently, Point-of-Care (POC) ECG monitoring works either as plot devices or alarms for abnormal cardiac rhythms using
predefined normal trigger ranges. Modern devices also incorporate algorithms to analyze arrhythmias improving their
specificity. On the other hand, full 12-derivation ECG machines are complex, are not suited to use as simple monitors
and are used with strict techniques for formal diagnostics of hearth electric conduction pathologies. The automatic
diagnostics are derived from a complete analysis of the 12-dimension data after it is fully and well collected. Both
systems do not handle disconnected leads and patient's motions, being strictly necessary to have a good and stable
signal to allow proper diagnosis. These interferences with the data collection frequently originate false alarms
increasing both patient and staff's stress; depending on how it is measured, the rate of false alarms (overall) in ICU
is estimated at 65 to 95\%\textsuperscript{\protect\hyperlink{ref-donchin2002}{1}}.

Alarm fatigue is a well-known problem that consists of a sensory overload of nurses and clinicians, resulting in
desensitization to alarms and missed alarms (the ``crying wolf'' situation). Patient deaths have been attributed to alarm
fatigue\textsuperscript{\protect\hyperlink{ref-sendelbach2013}{2}}. In 1982 it was recognized the increase in alarms with ``no end in sight''; studies have
demonstrated that most alarm signals have no clinical relevance and lead to clinical personnel's delayed response.
Ultimately patient deaths were reported related to inappropriate responses to alarms\textsuperscript{\protect\hyperlink{ref-sendelbach2013}{2}}.

In April of 2013, The Joint Commission\textsuperscript{\protect\hyperlink{ref-the_jc}{3}} issued the Sentinel Event Alert\textsuperscript{\protect\hyperlink{ref-JointCommission2013}{4}}, establishing
alarm system safety as a top hospital priority in the National Patient Safety Goal. Nowadays (2021), this subject still
on their list, in fourth place of importance\textsuperscript{\protect\hyperlink{ref-the_jc2021}{5}}.

In February of 2015, the CinC/Physionet Challenge 2015 was about "Reducing False Arrhythmia Alarms in the
ICU\textsuperscript{\protect\hyperlink{ref-Clifford2015}{6}}. The introduction article stated that it had been reported that up to 86\% resulting of the alarms are
false, and this can lead to decreased staff attention and an increase in patients' delirium\textsuperscript{\protect\hyperlink{ref-Lawless1994}{7}--\protect\hyperlink{ref-Parthasarathy2004}{9}}.

Due to this matter's importance, this research aims to identify abnormal hearth electric patterns using streaming data,
specifically those who are life-threatening, reducing the false alarms, being a reliable signal for Intensive Care Units
to respond quickly to those situations.

\hypertarget{objectives-and-the-research-question}{%
\chapter{Objectives and the research question}\label{objectives-and-the-research-question}}

This research aims to identify, on streaming data, abnormal hearth electric patterns, specifically those which are
life-threatening, to be a reliable signal for Intensive Care Units to respond quickly to those situations. It also may
be able to continuously analyze new data and correct itself shutting off false alarms.

As it is known, this goal is not a new problem, so the main questions to solve are: (1) Can we reduce the number of
false alarms in the ICU setting? (2) Can we accomplish this objective using a minimalist approach (low CPU, low memory)
while maintaining robustness? (3) Can this approach be used in other health domains other than ICU or ECG?

\hypertarget{related-works}{%
\chapter{Related Works}\label{related-works}}

The CinC/Physionet Challenge 2015 produced several papers aiming to reduce false alarms on their dataset. On Table
\ref{tab:alarms} it is listed the five life-threatening alarms present in their dataset.
\begin{table}

\caption{\label{tab:alarms}Definition of the 5 alarm types used in CinC/Physionet Challenge 2015 challenge.}
\centering
\begin{tabu} to \linewidth {>{\raggedright\arraybackslash}p{5cm}>{\raggedright}X}
\toprule
\textbf{Alarm} & \textbf{Definition}\\
\midrule
Asystole & No QRS for at least 4 seconds\\
Extreme Bradycardia & Heart rate lower than 40 bpm for 5 consecutive beats\\
Extreme Tachycardia & Heart rate higher than 140 bpm for 17 consecutive beats\\
Ventricular Tachycardia & 5 or more ventricular beats with heart rate higher than 100 bpm\\
Ventricular Flutter/Fibrillation & Fibrillatory, flutter, or oscillatory waveform for at least 4 seconds\\
\bottomrule
\end{tabu}
\end{table}
They used as score the following formula, which penalizes five times the false negatives (since we do not want to miss
any real event):

\[Score=\frac{TP+TN}{TP+TN+FP+5*FN}\]

The five-best scores in this challenge are presented on Table \ref{tab:scores}\textsuperscript{\protect\hyperlink{ref-plesinger2015}{10}--\protect\hyperlink{ref-hoogantink2015}{14}}.
\begin{table}

\caption{\label{tab:scores}Challenge Results on Streaming}
\centering
\begin{tabu} to \linewidth {>{\centering}X>{\raggedright\arraybackslash}p{9cm}}
\toprule
\textbf{Score} & \textbf{Authors}\\
\midrule
81.39 & Filip Plesinger, Petr Klimes, Josef Halamek, Pavel Jurak\\
79.44 & Vignesh Kalidas\\
79.02 & Paula Couto, Ruben Ramalho, Rui Rodrigues\\
76.11 & Sibylle Fallet, Sasan Yazdani, Jean-Marc Vesin\\
75.55 & Christoph Hoog Antink, Steffen Leonhardt\\
\bottomrule
\end{tabu}
\end{table}
Their algorithm did a pretty good job on the Physionet test-set. However, independently of their approach to this
problem, none of the authors reported benchmarks, memory usage, robustness test, or context invariance that could assure
its implementation on real monitors to reduce alarm fatigue indeed.

There are other arrhythmias that this challenge did not assess, like atrial standstill (hyperkalemia), third-degree
atrioventricular block, and others that may be life-threatening in some settings. Pulseless electrical activity is a
frequent condition in cardiac arrest but cannot be identified without blood pressure information. This information is
usually present in ICU settings but not in other locations.

\hypertarget{the-planned-approach-and-methods-for-solving-the-problem}{%
\chapter{The planned approach and methods for solving the problem}\label{the-planned-approach-and-methods-for-solving-the-problem}}

\hypertarget{state-of-the-art}{%
\section{State of the art}\label{state-of-the-art}}

A literature review of the last ten years is being conducted to assess state of the art for ECG automatic processing
collecting the following points if available : (1) The memory and space used to perform the primary goal of the
algorithm (sound an alarm, for ex.). (2) The type of algorithms used to identify ECG anomalies. (3) The type of
algorithms used to identify specific diagnoses (like a flutter, hyperkalemia, and others). (4) Their performance
(accuracy, ROC, etc.)

A broad search will be conducted on Pubmed, Scopus, Google Scholar, device manuals, and other specific sources.

Keywords:
\begin{itemize}
\tightlist
\item
  ECG AND monitoring AND ICU
\item
  ECG AND{[}time series{]}
\item
  ECG AND automatic AND interpretation
\end{itemize}
Articles published after ``The PhysioNet/Computing in Cardiology Challenge 2015: Reducing False Arrhythmia Alarms in the
ICU'' will also be analyzed.

\hypertarget{research-plan-and-methods}{%
\section{Research plan and methods}\label{research-plan-and-methods}}

This research is being conducted using the Research Compendium principles\textsuperscript{\protect\hyperlink{ref-compendium2019}{15}}:
\begin{enumerate}
\def\labelenumi{\arabic{enumi}.}
\tightlist
\item
  Stick with the convention of your peers;
\item
  Keep data, methods, and output separated;
\item
  Specify your computational environment as clearly as you can.
\end{enumerate}
Data management is following the FAIR principle (findable, accessible, interoperable, reusable)\textsuperscript{\protect\hyperlink{ref-wilkinson2016}{16}}.

Currently, the dataset used is stored on a public repository\textsuperscript{\protect\hyperlink{ref-franz_dataset}{17}}, the source code is publicly open and
stored on Github\textsuperscript{\protect\hyperlink{ref-franz_github}{18}}, while the reports and reproducibility information on each step is found on a public
website\textsuperscript{\protect\hyperlink{ref-franz_website}{19}}.

\hypertarget{type-of-study}{%
\subsection{Type of study}\label{type-of-study}}

This thesis will be a diagnostic study as the algorithm must classify the change in pattern as positive or negative for
life-threatening.

\hypertarget{the-data}{%
\subsection{The data}\label{the-data}}

Initially we will use the CinC/Physionet Challenge 2015 dataset that is publicly available on Physionet. This dataset is
a good start for exploring the main goal of reduce false alarms. This dataset was manually selected for this challenge
and the events were labeled by experts, so it is not RAW data. All signals have been resampled (using anti-alias
filters) to 12 bit, 250 Hz and have had FIR bandpass {[}0.05 to 40Hz{]} and mains notch filters applied to remove noise.
Pacemaker and other artifacts may be present on the ECG\textsuperscript{\protect\hyperlink{ref-Clifford2015}{6}}. Furthermore, this dataset contains at least two
ECG derivations and one or more variables like arterial blood pressure, photoplethysmograph readings, and respiration
movements.

These variables may or may not be helpful for increasing the sensitivity or specificity of the algorithm. It is planned
to use the minimum set of variables as it is known in multi-dimensional analysis that using just two (or some small
subset) of all the dimensions can be much more accurate than either using all dimensions or a single
dimension\textsuperscript{\protect\hyperlink{ref-gharghabi2018}{20}}.

It is desirable that real data extracted from Portuguese ICU could be used in further stage to assess the validity of
the model in real settings and robustness (using RAW data instead of filtered data). The variables available on
Physionet's dataset are commonly available on Portuguese ICU's.

\hypertarget{workflow}{%
\subsection{Workflow}\label{workflow}}

All steps of the process will be managed using the R package \texttt{targets}\textsuperscript{\protect\hyperlink{ref-landau2021}{21}} from data extraction to the final
report, as shown in Fig. \ref{fig:targets}.

The report will then be available on the main webpage\textsuperscript{\protect\hyperlink{ref-franz_website}{19}}, allowing inspection of previous versions managed
by the R package \texttt{workflowr}\textsuperscript{\protect\hyperlink{ref-workflowr2021}{22}}, as we can see in Fig. \ref{fig:workflow_workflowr}.

\hypertarget{statistical-analysis}{%
\subsection{Statistical analysis}\label{statistical-analysis}}

The Statistical analysis will be performed using R language v4.0.4 or greater and it will be computed the ROC curve for
the algorithm.

The experiment will be conducted using the Matrix Profile concept\textsuperscript{\protect\hyperlink{ref-yeh2016}{23}}, the state-of-the-art for time series
analysis. It will be conducted several experiments to identify the best algorithm for this task. One of such algorithms
is the online semantic segmentation for multi-dimensional time series\textsuperscript{\protect\hyperlink{ref-gharghabi2018}{20}}.

In addition, we will combine the fading factors\textsuperscript{\protect\hyperlink{ref-Gama2013}{24},\protect\hyperlink{ref-Rodrigues2010}{25}} strategy to minimize the memory and space
consumption lowering the processor power needed, allowing this algorithm to be used in almost any device.

\hypertarget{research-team}{%
\subsection{Research Team}\label{research-team}}
\begin{itemize}
\tightlist
\item
  Thesis Author: Francisco Bischoff
\item
  Supervisor: Professor Pedro Pereira Rodrigues
\item
  Co-supervisor: Professor Eamonn Keogh (UCR, Riverside)
\end{itemize}
\hypertarget{expected-results-and-outcomes}{%
\section{Expected results and outcomes}\label{expected-results-and-outcomes}}

We expect the following results: (1) Identify the obstacles of identifying life-threatening ECG changes within memory,
space, and CPU constraints. (2) Be able to reduce ECG monitor's false alarms using the proposed methodology. (3) Assess
the feasibility of implementing the algorithm in the real world and other settings than ICU monitors.

And outcomes: (1) To achieve a better score of false alarm reduction than the best on Physionet's 2015 challenge. (2) To
push forward the state-of-the-art technology on false alarms reduction, maybe even being domain agnostic. (3) To draw
more attention to fading factors as a reliable, fast, and cheap approximation of the true value. (4) To draw more
attention to the matrix profile concept as an efficient, agnostic, and almost parameter-free way to analyze time series.
(5) To draw more attention of the Patient Monitorization industry on solving the false alarm problem.

\hypertarget{whatever}{%
\chapter{Whatever}\label{whatever}}

\textbf{Research question and aims}

This research aims to identify, on streaming data, abnormal hearth electric patterns, specifically those who are
life-threatening, in order to be a reliable signal for Intensive Care Units to respond quickly to those situations. It
also may be able to continuously analyze new data and correct itself shutting off false alarms. Primarily an experiment
will be conducted using 2 main algorithms that use Matrix Profile in detecting context changes: SDTD and FLOSS. One uses
whole data training and testing, and the other uses a streaming approach that is our main interest. The goal will be
detecting the transition from normal to flutter/FA to normal condition with special attention to not rely on rhythm
changes. Being this successful, a more generalistic approach will be attempted: to detect changes from normal to
abnormal to normal conditions, with special attention to handle with disconnected leads or patient movements. Finally,
this research can prove to be a good addition to the Matrix Profile method, using fading factors in order to reduce
memory and space consumption, lowering the processor power needed, allowing this algorithm to be used in almost any
device.

\textbf{About the ongoing project}

The document submitted for approval is
\href{https://github.com/franzbischoff/false.alarm/blob/master/protocol/Protocol.pdf}{here}.

The full code is open-sourced and available \href{https://github.com/franzbischoff/false.alarm/}{here}

To follow the thesis timeline you can access the full Gantt chart at Zenhub. Click
\href{https://app.zenhub.com/workspaces/phd-thesis-5eb2ce34f5f30b3aed0a35af/roadmap}{here} (you need a github account, but
that's it).

PDF, EPUB and WORD versions will be available at the end of this work.

\backmatter

\hypertarget{references}{%
\chapter*{References}\label{references}}
\addcontentsline{toc}{chapter}{References}

\markboth{References}{References}

\noindent

\setlength{\parindent}{-0.20in}
\setlength{\leftskip}{0.20in}
\setlength{\parskip}{8pt}

\hypertarget{refs}{}
\begin{CSLReferences}{0}{0}
\leavevmode\vadjust pre{\hypertarget{ref-donchin2002}{}}%
\CSLLeftMargin{1. }
\CSLRightInline{Donchin Y, Seagull FJ. The hostile environment of the intensive care unit. \emph{Current Opinion in Critical Care}. 2002;8(4):316-320. doi:\href{https://doi.org/10.1097/00075198-200208000-00008}{10.1097/00075198-200208000-00008}}

\leavevmode\vadjust pre{\hypertarget{ref-sendelbach2013}{}}%
\CSLLeftMargin{2. }
\CSLRightInline{Sendelbach S, Funk M. Alarm Fatigue. \emph{AACN Advanced Critical Care}. 2013;24(4):378-386. doi:\href{https://doi.org/10.4037/nci.0b013e3182a903f9}{10.4037/nci.0b013e3182a903f9}}

\leavevmode\vadjust pre{\hypertarget{ref-the_jc}{}}%
\CSLLeftMargin{3. }
\CSLRightInline{The joint commission. Published 2021. Accessed April 8, 2021. \url{https://www.jointcommission.org}}

\leavevmode\vadjust pre{\hypertarget{ref-JointCommission2013}{}}%
\CSLLeftMargin{4. }
\CSLRightInline{Joint Commission. {Sentinel event alert - Medical device alarm safety in hospitals.} 2013;(50):1-3.}

\leavevmode\vadjust pre{\hypertarget{ref-the_jc2021}{}}%
\CSLLeftMargin{5. }
\CSLRightInline{The joint commission - national patient safety goals. Published 2021. Accessed April 8, 2021. \url{https://www.jointcommission.org/standards/national-patient-safety-goals/hospital-national-patient-safety-goals/}}

\leavevmode\vadjust pre{\hypertarget{ref-Clifford2015}{}}%
\CSLLeftMargin{6. }
\CSLRightInline{Clifford GD, Silva I, Moody B, et al. The PhysioNet/computing in cardiology challenge 2015: Reducing false arrhythmia alarms in the ICU. In: \emph{Computing in Cardiology}.; 2015. doi:\href{https://doi.org/10.1109/cic.2015.7408639}{10.1109/cic.2015.7408639}}

\leavevmode\vadjust pre{\hypertarget{ref-Lawless1994}{}}%
\CSLLeftMargin{7. }
\CSLRightInline{Lawless ST. Crying wolf: False alarms in a pediatric intensive care unit. \emph{Critical care medicine}. 1994;22(6):981-985.}

\leavevmode\vadjust pre{\hypertarget{ref-Chambrin2001}{}}%
\CSLLeftMargin{8. }
\CSLRightInline{Chambrin MC. Alarms in the intensive care unit: How can the number of false alarms be reduced? \emph{Critical care (London, England)}. 2001;5(4):184-188. doi:\href{https://doi.org/10.1186/cc1021}{10.1186/cc1021}}

\leavevmode\vadjust pre{\hypertarget{ref-Parthasarathy2004}{}}%
\CSLLeftMargin{9. }
\CSLRightInline{Parthasarathy S, Tobin MJ. Sleep in the intensive care unit. \emph{Intensive Care Medicine}. 2004;30(2):197-206. doi:\href{https://doi.org/10.1007/s00134-003-2030-6}{10.1007/s00134-003-2030-6}}

\leavevmode\vadjust pre{\hypertarget{ref-plesinger2015}{}}%
\CSLLeftMargin{10. }
\CSLRightInline{Plesinger F, Klimes P, Halamek J, Jurak P. 2015 computing in cardiology conference (CinC). In: IEEE; 2015. doi:\href{https://doi.org/10.1109/cic.2015.7408641}{10.1109/cic.2015.7408641}}

\leavevmode\vadjust pre{\hypertarget{ref-kalidas2015}{}}%
\CSLLeftMargin{11. }
\CSLRightInline{Kalidas V, Tamil LS. 2015 computing in cardiology conference (CinC). In: IEEE; 2015. doi:\href{https://doi.org/10.1109/cic.2015.7411015}{10.1109/cic.2015.7411015}}

\leavevmode\vadjust pre{\hypertarget{ref-couto2015}{}}%
\CSLLeftMargin{12. }
\CSLRightInline{Couto P, Ramalho R, Rodrigues R. 2015 computing in cardiology conference (CinC). In: IEEE; 2015. doi:\href{https://doi.org/10.1109/cic.2015.7411019}{10.1109/cic.2015.7411019}}

\leavevmode\vadjust pre{\hypertarget{ref-fallet2015}{}}%
\CSLLeftMargin{13. }
\CSLRightInline{Fallet S, Yazdani S, Vesin J-M. 2015 computing in cardiology conference (CinC). In: IEEE; 2015. doi:\href{https://doi.org/10.1109/cic.2015.7408640}{10.1109/cic.2015.7408640}}

\leavevmode\vadjust pre{\hypertarget{ref-hoogantink2015}{}}%
\CSLLeftMargin{14. }
\CSLRightInline{Hoog Antink C, Leonhardt S. 2015 computing in cardiology conference (CinC). In: IEEE; 2015. doi:\href{https://doi.org/10.1109/cic.2015.7408642}{10.1109/cic.2015.7408642}}

\leavevmode\vadjust pre{\hypertarget{ref-compendium2019}{}}%
\CSLLeftMargin{15. }
\CSLRightInline{Research compendium. Published 2019. Accessed April 8, 2021. \url{https://research-compendium.science}}

\leavevmode\vadjust pre{\hypertarget{ref-wilkinson2016}{}}%
\CSLLeftMargin{16. }
\CSLRightInline{Wilkinson MD, Dumontier M, Aalbersberg IjJ, et al. The FAIR Guiding Principles for scientific data management and stewardship. \emph{Scientific Data}. 2016;3(1). doi:\href{https://doi.org/10.1038/sdata.2016.18}{10.1038/sdata.2016.18}}

\leavevmode\vadjust pre{\hypertarget{ref-franz_dataset}{}}%
\CSLLeftMargin{17. }
\CSLRightInline{The PhysioNet computing in cardiology challenge 2015 - dataset. Published 2021. Accessed April 8, 2021. \url{https://zenodo.org/record/4634014}}

\leavevmode\vadjust pre{\hypertarget{ref-franz_github}{}}%
\CSLLeftMargin{18. }
\CSLRightInline{Franzbischoff/false.alarm: PhD programme in health data science. Published 2020. Accessed April 8, 2021. \url{https://github.com/franzbischoff/false.alarm}}

\leavevmode\vadjust pre{\hypertarget{ref-franz_website}{}}%
\CSLLeftMargin{19. }
\CSLRightInline{Franzbischoff/false.alarm: Reproducible reports. Published 2021. Accessed April 8, 2021. \url{https://franzbischoff.github.io/false.alarm}}

\leavevmode\vadjust pre{\hypertarget{ref-gharghabi2018}{}}%
\CSLLeftMargin{20. }
\CSLRightInline{Gharghabi S, Yeh C-CM, Ding Y, et al. Domain agnostic online semantic segmentation for multi-dimensional time series. \emph{Data Mining and Knowledge Discovery}. 2018;33(1):96-130. doi:\href{https://doi.org/10.1007/s10618-018-0589-3}{10.1007/s10618-018-0589-3}}

\leavevmode\vadjust pre{\hypertarget{ref-landau2021}{}}%
\CSLLeftMargin{21. }
\CSLRightInline{Landau W, Landau W, Warkentin MT, et al. \emph{Ropensci/Targets, Dynamic Function-Oriented 'Make'-Like Declarative Workflows}. Zenodo; 2021. doi:\href{https://doi.org/10.5281/ZENODO.4062936}{10.5281/ZENODO.4062936}}

\leavevmode\vadjust pre{\hypertarget{ref-workflowr2021}{}}%
\CSLLeftMargin{22. }
\CSLRightInline{Blischak JD, Carbonetto P, Stephens M. Creating and sharing reproducible research code the workflowr way {[}version 1; peer review: 3 approved{]}. \emph{F1000Research}. 2019;8(1749). doi:\href{https://doi.org/10.12688/f1000research.20843.1}{10.12688/f1000research.20843.1}}

\leavevmode\vadjust pre{\hypertarget{ref-yeh2016}{}}%
\CSLLeftMargin{23. }
\CSLRightInline{Yeh C-CM, Zhu Y, Ulanova L, et al. 2016 IEEE 16th international conference on data mining (ICDM). In: IEEE; 2016. doi:\href{https://doi.org/10.1109/icdm.2016.0179}{10.1109/icdm.2016.0179}}

\leavevmode\vadjust pre{\hypertarget{ref-Gama2013}{}}%
\CSLLeftMargin{24. }
\CSLRightInline{Gama J, Sebastião R, Rodrigues PP. {On evaluating stream learning algorithms}. \emph{Machine Learning}. 2013;90(3):317-346. doi:\href{https://doi.org/10.1007/s10994-012-5320-9}{10.1007/s10994-012-5320-9}}

\leavevmode\vadjust pre{\hypertarget{ref-Rodrigues2010}{}}%
\CSLLeftMargin{25. }
\CSLRightInline{Rodrigues PP, Gama J, Sebastião R. {Memoryless Fading Windows in Ubiquitous Settings}. In: \emph{Proceedings of the First Ubiquitous Data Mining Workshop}.; 2010:23-27.}

\end{CSLReferences}

% Index?

\end{document}
