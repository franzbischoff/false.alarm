% This is samplepaper.tex, a sample chapter demonstrating the
% LLNCS macro package for Springer Computer Science proceedings;
% Version 2.20 of 2017/10/04
%
\documentclass[runningheads]{llncs}
%
\usepackage{amsmath}
\usepackage{graphicx}
\usepackage[utf8]{inputenc}
\usepackage[hyphens]{url} % not crucial - just used below for the URL
\usepackage{hyperref}
\hypersetup{
    colorlinks=true,
    linkcolor=blue,
    filecolor=blue,
    urlcolor=blue,
}
\providecommand{\tightlist}{%
  \setlength{\itemsep}{0pt}\setlength{\parskip}{0pt}}
% Used for displaying a sample figure. If possible, figure files should
% be included in EPS format.

% Pandoc citation processing
$if(csl-refs)$
\newlength{\cslhangindent}
\setlength{\cslhangindent}{1.5em}
\newlength{\csllabelwidth}
\setlength{\csllabelwidth}{3em}
\newenvironment{CSLReferences}[2] % #1 hanging-ident, #2 entry spacing
 {% don't indent paragraphs
  \setlength{\parindent}{0pt}
  % turn on hanging indent if param 1 is 1
  \ifodd #1 \everypar{\setlength{\hangindent}{\cslhangindent}}\ignorespaces\fi
  % set entry spacing
  \ifnum #2 > 0
  \setlength{\parskip}{#2\baselineskip}
  \fi
 }%
 {}
\usepackage{calc}
\newcommand{\CSLBlock}[1]{#1\hfill\break}
\newcommand{\CSLLeftMargin}[1]{\parbox[t]{\csllabelwidth}{#1}}
\newcommand{\CSLRightInline}[1]{\parbox[t]{\linewidth - \csllabelwidth}{#1}\break}
\newcommand{\CSLIndent}[1]{\hspace{\cslhangindent}#1}
$endif$

\begin{document}
%
\title{$title$\thanks{$thanks$}}
%
%\titlerunning{Abbreviated paper title}
% If the paper title is too long for the running head, you can set
% an abbreviated paper title here
%
%
\author{ $for(authors)$ $it.name$\inst{$it.inst$}\orcidID{$it.orcid$}${sep}, $endfor$ }

$if(authorrunning)$
\authorrunning{ $authorrunning$ }
$endif$
% First names are abbreviated in the running head.
% If there are more than two authors, 'et al.' is used.
%
\institute{ $for(institutes)$ \textsuperscript{$it.num$} $it.dept$${sep}\\ $endfor$ }

% \institute{$institutes$}

\maketitle              % typeset the header of the contribution
%

\begin{abstract}
  $abstract$

  \keywords{
    $for(keywords)$
    $keywords$${sep} \and
    $endfor$
  }

\end{abstract}
%
%
% \def\spacingset#1{\renewcommand{\baselinestretch}%
% {#1}\small\normalsize} \spacingset{1}

$body$

\end{document}
